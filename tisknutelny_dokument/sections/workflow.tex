%%%%%%%%%%%%%%%%%%%%%%%%%%%%%%%%%%%%%%%%%%%%%%%%%%%%%%%%%%%%%%%%%%%%%%%%%%%%%%%
%%%%%%%%%%%%%%%%%%%%%%%%%%%%%%%%%%%%%%%%%%%%%%%%%%%%%%%%%%%%%%%%%%%%%%%%%%%%%%%
%%%%%%%%%%%%%%%%%%%%%%%%%%%%%%%%%%%%%%%%%%%%%%%%%%%%%%%%%%%%%%%%%%%%%%%%%%%%%%%

\section{Workflow získání a zpracování textu}

Zpracování\index{analýza} textu přirozeného jazyka patří mezi úlohy oblasti
\textit{\underline{n}atural \underline{l}anguage \underline{p}rocessing}
(NLP)\index{zpracování (processing)!natural language processing}, kam lze v širším slova smyslu
řadit kromě zpracování textů
i práci s netextovými záznamy daného jazyka.

My se zde omezíme pouze na práci s textovými zdroji daného přirozeného
jazyka, konkrétně na jako reprezentaci formou rozsáhlého souboru
(ne nutně konzistentních) vět psaných v daném jazyce, tedy formou tzv.
\textit{korpusu}\index{korpus}.

Jednotlivé procedury zpracování textu přirozeného jazyka na sebe navazují,
jak ukazuje obrázek~\ref{pipeline_fig}. 


\begin{figure}[h]
\centering
\begin{tikzpicture}[node distance=1cm, auto,]
  \node[punkt] (helsinki_corpora) {výchozí korpus};
  \node[punkt, right=1.0cm of helsinki_corpora] (webscraping) {%
    obohacení korpusu webscrapingem
  }
    edge[pil, bend left=45] (helsinki_corpora);
  \node[punkt, below=of helsinki_corpora] (sentence_splitting) {%
    rozdělení korpusu na věty
  };
  \node[punkt, opacity=0, text opacity=0] (helsinki_corpora_dummy) {%
    výchozí korpus
  }
    edge[pil, bend left=0] (sentence_splitting);
  \node[punkt, below=of sentence_splitting] (text_processing) {%
    předzpracování textu
  };
  \node[punkt, below=of helsinki_corpora, opacity=0, text opacity=0] (%
    sentence_splitting_dummy) {rozdělení korpusu na věty}
    edge[pil, bend left=0] (text_processing);
  \node[punkt, below=of text_processing] (tokenization) {%
    tokenizace vět na slova
  };
  \node[punkt, below=of sentence_splitting, opacity=0, text opacity=0] (%
  text_processing_dummy) {předzpracování textu}
    edge[pil, bend left=0] (tokenization);
  \node[punkt, below=of tokenization] (words_removing) {%
    odstranění stop slov a~vulgarismů,
    číslic a~krátkých slov
  };
  \node[punkt, below=of text_processing, opacity=0, text opacity=0] (%
    tokenization_dummy) {tokenizace vět na slova}
    edge[pil, bend left=0] (words_removing);
  \node[punkt, below=of words_removing] (n_gramming) {$n$-gramming};
  \node[punkt, below=of tokenization, opacity=0, text opacity=0] (%
    words_removing_dummy%
  ) {odstranění stop slov a~vulgarismů, číslic a~krátkých slov}
    edge[pil, bend left=0] (n_gramming);
\end{tikzpicture}
\caption{Schéma (zjednodušené) asynchronních úloh během zpracování textu %
         pro účely jeho statistické reprezentace ($n$-grammingu)%
         \index{$n$-gram}%
         \label{pipeline_fig}}
\end{figure}


V tomto pojednání se zaměřujeme na
zpracování textů psaných anglicky, neboť angličtina\index{angličtina} je
analytický
jazyk~\cite{Gorblach1991}\index{jazyk!analytický}, jehož textové a
jazykově-statistické zpracování
je při dnešních možnostech relativně dobře dobře zvládnutelné, a to
především díky minimálnímu zatížení anglických textů morfematikou%
\index{morfematika}
ohebných slovních druhů; \textit{lemmatizaci}\index{lemmatizace} slov, tedy
jejich převod
na morfologicky základní tvar, lze v anglickém textu
v podstatě vynechat, aniž by byla ohrožena kvalita analýzy textu i konečného
$n$-grammingu\footnote{Opakem by byla čeština, tedy flekční jazyk%
\index{jazyk!flekční}, ve kterém
je jazyková analýza prakticky vždy závislá na kvalitě lemmatizace%
\index{lemmatizace} kvůli bohaté
morfologii s relativně vysokou mírou nepatternových výjimek.} Proto ve
schématu na obrázku~\ref{pipeline_fig} chybí fáze lemmatizace.

Výchozím bodem je stav, kdy je k dispozici existující korpus\index{korpus}
anglických textů
(v~následujících statích bude zpracovávána část tzv.
\textit{Helsinki corpora}~\cite{Kytoo1996}\index{korpus!helsinský};
jde o anglické texty různého
původu -- pocházejí z anglických zpravodajských relací, blogů
a~z~twitteru). K nim byl
přidán ještě menší korpus získaný vlastním webscrapingem\index{webscraping}
malé části
anglicky mluvící Wikipedie\index{Wikipedie}.

Korpusy je poté možné dohromady jednotně předzpracovat.
Bloky textů korpusů jsou nejdříve rozděleny na věty, a to podle určité
heuristiky (viz dále).

Dále je text tvořený větami v rámci předzpracování\index{zpracování (processing)!předzpracování (preprocessing)} očištěn
o některé nadbytečné znaky (především interpunkci\index{interpunkce});
v závěru této fáze jsou korpusy\index{korpus} tvořeny
oddělenými větami, které tvoří písmena malé a velké anglické abecedy
a mezery. Velká písmena byla v rámci redukce "variability" textu převedena
bez ztráty informace na malá písmena. Rovněž je v této fázi žádoucí
oprostit slova vět o ty, které nesou pouze kontextovou informaci, zde jde
především o číslice. Některá velmi krátká slova (obvykle o délce kratší
než tři písmena) lze považovat za výsledky odstranění interpunkce%
\index{interpunkce} (zejména
apostrofu v rámci zkratkovým forem, např. kdy z "i'm" jakožto původně
tříznakového "slova" zbude pouze "i" a "m"), kdy obvykle vznikají jen
oddělené zkrácené části fráze, které má smysl také odstranit.

Následuje fáze tokenizace\index{tokenizace} vět na jednotlivá slova,
výsledkem je pro daný korpus množina uspořádaných, různě dlouhých
$k$-tic slov, kde $k \geq 1$ je pro každou větu nějaké přirozené číslo
označující počet slov této věty. V této fázi je možné z vět odstranit ta
slova, která v textech nejsou z nějakého smysluplného důvodu žádoucí.
Jde především o vulgarismy\index{slova!vulgarismy} a jinak nevhodná
slova\index{slova!nevhodná}, dále je
možné (ale ne nutné) odstranit tzv. stop slova\index{slova!stop slova}
-- jde obvykle o často
se objevující slova, většinou neohebných slovních druhů (není pravidlem),
která nenesou příliš zajímavou informaci.

Nyní by v případě flekčního jazyka\index{jazyk!flekční} následovala fáze
lemmatizace\index{lemmatizace}, tedy snížení
variability slov (konkrétně ohebných slovních druhů) jejich převodem vždy na základní
tvar. Tato podúloha je však netriviální, naštěstí v případě anglického jazyka%
\index{angličtina},
jak již bylo zmíněno, ji lze vynechat bez významného snížení kvality
předzpracování jazyka. Morfologie\index{morfematika} anglických slov zahrnuje
v podstatě jen
přidání koncovek "-s" či "-es" v~případě plurálu.

Následně, když jsou korpusy zpracovány do té fáze, že jde o velké
množství uspořádaných $k$-tic vždy pro přirozená $k \geq 1$ lišící se
mezi původními větami, je možné text korpusů uchopit z pohledu
statistického modelování jazyka a nalézt v~textu tzv. $n$-gramy\index{$n$-gram}
pro malá
$n$, typicky $n \in \{2, 3, 4, 5\}$, ale i vyšší.
Jde o $n$-členná sousloví, tedy $n$-tice sousedících, po sobě
jdoucích slov\cite{Manning1999}. Po sestavení $n$-gramů\index{$n$-gram} je možné vytvořit
tabulku četností pro jednotlivá $n$-členná slovní spojení a získat tak bodové
odhady apriorních pravděpodobností, s jakými se v textu daného
přirozeného jazyka vyskytují.

Takové tabulky četností\index{$n$-gram!tabulka četností}
$n$-gramů\index{$n$-gram} pro $n \in \{2, 3, 4, 5\}$ či vyšší
jsou základem pro algoritmy\index{algoritmus} předpovědí takového slova, které by v textu
daného přirozeného jazyka následovalo s největší pravděpodobností po
zadaných $n - 1$ slovech tvořících jazykové sousloví.

K tomuto účelu byla vytvořena webová aplikace
\texttt{the\_next\_word\_prediction}\index{aplikace!\texttt{the\_next\_word\_prediction}},
která uživateli po zadání
$(n - 1)$-členného slovního spojení v daném jazyce vrátí
nejpravděpodobněji následující $n$-té slovo\index{$n$-té slovo}. Její
principy fungování budou rovněž popsány.


\newpage


%%%%%%%%%%%%%%%%%%%%%%%%%%%%%%%%%%%%%%%%%%%%%%%%%%%%%%%%%%%%%%%%%%%%%%%%%%%%%%%
%%%%%%%%%%%%%%%%%%%%%%%%%%%%%%%%%%%%%%%%%%%%%%%%%%%%%%%%%%%%%%%%%%%%%%%%%%%%%%%
%%%%%%%%%%%%%%%%%%%%%%%%%%%%%%%%%%%%%%%%%%%%%%%%%%%%%%%%%%%%%%%%%%%%%%%%%%%%%%%





